\documentclass[a4paper]{report}

\def\baselinestretch{1.1}
\usepackage[14pt]{extsizes}
\usepackage[utf8]{inputenc}
\usepackage[russian]{babel}
\usepackage{indentfirst}
\usepackage{mathrsfs}
\usepackage{graphicx}
\usepackage{float}
\usepackage{wrapfig}

%%%%%%%%%%%%%%%%% Символы, графика %%%%%%%%%%%%%%%%%%%%%

\usepackage[T2A]{fontenc}
\usepackage{amsmath,amssymb,amsfonts,amsthm}
\usepackage{bm}
\usepackage{graphicx}
\usepackage{color}
\usepackage[pdftex,colorlinks,linkcolor=blue,citecolor=blue]{hyperref}
\usepackage{pgfplots}
\usepackage{tikz}

%%%%%%%%% Разметка страницы %%%%%%%%%

\usepackage{indentfirst}
\topmargin=-1.5cm %отступ сверху
\oddsidemargin=0.4cm %отступ слева (нечетные страницы)
\evensidemargin=0.4cm %(четные страницы)
\textwidth=16cm %ширина текста
\textheight=24cm
\tolerance=800
\parskip=1ex

\pagestyle{plain}

\usepackage{listings}

\definecolor{codegreen}{rgb}{0,0.6,0}
\definecolor{codegray}{rgb}{0.5,0.5,0.5}
\definecolor{codepurple}{rgb}{0.58,0,0.82}
\definecolor{backcolour}{rgb}{0.95,0.95,0.92}

\lstset{
    backgroundcolor=\color{backcolour},
    commentstyle=\color{codegreen},
    keywordstyle=\color{magenta},
    numberstyle=\tiny\color{codegray},
    stringstyle=\color{codepurple},
    breakatwhitespace=false,
    breaklines=true,
    captionpos=b,
    keepspaces=true,
    numbers=left,
    numbersep=3pt,
    showspaces=false,
    showstringspaces=false,
    showtabs=false,
    tabsize=1,
    basicstyle=\fontsize{10}{12}\selectfont\ttfamily
}

\newcommand{\ket}[1] {\!\!\;\ensuremath{\left|#1\right\rangle}}
\newcommand{\bra}[1] {\!\!\:\ensuremath{\left\langle#1\right|\!\!\:}}
\newcommand{\ketbra}[2]{\!\!\:\ensuremath {\left|#1\right\rangle\!\:\!\!\left\langle#2\right|}}
\newcommand{\braket}[2]{\ensuremath {\!\!\:\left\langle#1\!\!\: \left|\!\!\!\;\right.#2\right\rangle\!\!\;}}

\begin{document}

\begin{titlepage}
	\begin{center}
		Министерство науки и высшего образования РФ\\
		ФГБОУ ВО «Тверской государственный университет»\\
		Математический факультет\\
		Направление 02.04.01 Математика и компьютерные науки\\
		Профиль <<Математическое и компьютерное моделирование>>	
	\end{center}
	
	\vspace{2.5cm}
	\begin{center}
	
		{МАГИСТЕРСКАЯ ДИССЕРТАЦИЯ}
		
		
		\vspace{1.0cm}
		\large{Математические модели малоразмерных систем кубитов с разреженными гамильтонианами}
		
		
		\vspace{1.0cm}
	\end{center}
	
	
	
	\begin{flushright}
		\begin{minipage}{80mm}
			Автор:\\
			Шантуев Кирилл Дмитриевич
			
			\vspace{1.0cm}
			Научный руководитель:\\
			д. ф.-м. н. Цирулёв А.Н.
			
		\end{minipage}
	\end{flushright}
	
	
	\vspace{1.3cm}
	\noindent Допущен к защите:\\
	Руководитель ООП:\\[1cm]
	\underline{\qquad \qquad \qquad \qquad \qquad }
    В.П. Цветков \\
	\vspace{2.3cm}
	
	
	
	\begin{center}
		Тверь 2025
	\end{center}
	
	\date{}
\end{titlepage}

\setcounter{page}{2}

\tableofcontents
\newpage

% Abstract
\addcontentsline{toc}{chapter}{\hspace{7mm} Введение}

\section*{Введение}


Применение им существует в вариационных квантовых алгоритмах, которые подразумевают моделирование системы кубитов с гамильтонианом, который очень эффективен для данной реальной системы и гораздо проще её, при этом отражая все необходимые и присущие задаче характеристики.

Про малоразмерные гамильтонианы не забывают и в физике конденсированного состояния, где для таких систем используют разложение гамильтонианов, и в квантовой химии, в вопросе фермионных систем.

В конкретных вычислениях будет использоваться система атомных единиц Хартри, в которой постоянная Планка, масса электрона и заряд электрона выбираются в качестве основных единиц момента импульса, массы и заряда. Другими словами, принимается, что $\hbar=e=m_e=1$, а единица энергии обозначается ${1\mathrm{Ha}\approx4.36\!\cdot\!10^{-11}}$эрг. В этих единицах коэффициенты в разложении гамильтонианов по базису Паули будут порядка единицы и для электронных оболочек молекул, и для современных физических реализаций систем кубитов.

%####################################################
%################# Глава 1 ##########################
%####################################################

\chapter{Гильбертово пространство малоразмерной системы \\кубитов}

%####################################################

\section{Базисы и измерения}\label{BPauli}

Базисы и измерения являются основными концепциями в квантовой механике, которые помогают описывать поведение квантовых систем. Различные виды базисов определяют, как мы можем представлять состояния системы, а процесс измерения показывает, как эти состояния могут изменяться в результате взаимодействия с окружающим миром. Базисом гильбертова пространства принято называть систему линейно независимых и порождающих всё пространство векторов. Базисы в гильбертовом пространстве играют ключевую роль в квантовой механике и функциональном анализе. Гильбертово пространство --- это абстрактное комплексное линейное пространство, в котором определено эрмитово скалярное произведение. Оно может быть конечномерным или бесконечномерным, причем в последнем случае в определение включается требование полноты. Примерами могут служить бесконечномерное пространство квадратично интегрируемых функций ${L_2}$ и произвольное конечномерное пространство векторов; в конечномерном случае любые два гильбертовых пространства одинаковой размерности изоморфны и математически неотличимы, если только в них не определены дополнительные структуры. Важно отметить, что векторы гильбертовы пространства сами по себе ненаблюдаемы.

Всюду ниже рассматриваются только конечномерные (дуальные друг другу) гильбертовы пространства $\mathcal{H}_n$ и $\mathcal{H}_n^\dag$ размерности $N=2^n$, которые описывают состояния квантовой системы, состоящей из $n$ кубитов. Состояния системы --- комплексные прямые в $\mathcal{H}_n$ или $\mathcal{H}_n^\dag$ --- представляются векторами состояния, которые нормированы на единицу и определены с точностью до фазового множителя. Простейшее гильбертово пространство $\mathcal{H}$ размерности два представляет кубит, в котором стандартный ортонормированный базис в обозначениях Дирака обозначается как $\{\ket{0},\ket{1}\}$, где скалярные произведения равны
\begin{equation}\label{}
\braket{0}{0}=\braket{1}{1}=1, \qquad \braket{0}{1}=\braket{1}{0}=0.
\nonumber
\end{equation}

Любой вектор состояния в $\mathcal{H}$ или $\mathcal{H}^\dag$ можно представить разложением
\begin{equation}\label{}
\ket{u}=a\ket{0}+b\ket{1},\quad \bra{u}=\bar{a}\bra{0}+\bar{b}\bra{1},\qquad \braket{u}{u}=|a|^2+|b|^2=1.
\nonumber
\end{equation}

Гильбертово пространство системы $n$ взаимодействующих кубитов --- это тензорное произведение ($n$ сомножителей)
\begin{equation*}
\mathcal{H}_n= \mathcal{H}\otimes\cdots\otimes\mathcal{H},\qquad \mathcal{H}_n^\dag= \mathcal{H}^\dag\otimes\cdots\otimes \mathcal{H}^\dag.
\end{equation*}
Соответственно, стандартный ортонормированный базис в $\mathcal{H}_n$ является тензорным произведением элементов однокубитного базиса $\{\ket{0},\ket{1}\}$, т.е. это базис
\begin{equation*}
\left\{\vphantom{A^k_k}\ket{k}\right\}_{k=0}^{N-1},\quad \ket{k}= \ket{k_1\ldots k_n}= \ket{k_1}\otimes\ldots\otimes \ket{k_n},\quad k_1,\ldots,k_n\in\{0,1\},
\end{equation*}
где $k$ --- десятичное представление двоичной строки $k_1\ldots k_n$. Этот базис автоматически является ортонормированным: $\braket{k}{l}=\delta_{kl}$. Разложение произвольного вектора $\ket{u}\in\mathcal{H}_n$ имеет вид
\begin{equation}\label{}
\ket{u}\;=\;\; \sum\limits_{k=0}^{N-1} u_{k}\ket{k}= \sum\limits_{k_1,\ldots,k_n\in\{0,1\}} u_{k_1\ldots k_n}\ket{k_1\ldots k_n},
\nonumber
\end{equation}
где ${u_{k}= u_{k_1\ldots k_n}}$ - комплексные коэффициенты.

В квантовой теории измерения связаны с эрмитовыми операторами. Они образуют вещественное подпространство в комплексном линейном пространстве операторов $\mathcal{L}(\mathcal{H}_n)= \mathcal{H}_n\otimes\mathcal{H}_n^\dag$. В физической литературе эрмитовы операторы называют наблюдаемыми.

Для оператора ${\hat{A}\in\mathcal{L}(\mathcal{H}_n)}$ эрмитово сопряженным к нему называется оператор ${\hat{A}^\dag}$ такой, что
\begin{equation}\label{}
\bra{u}\hat{A}^\dag\ket{w}\,=\, \overline{\bra{w}\hat{A}\ket{u}},
\nonumber
\end{equation}
а если ${\hat{A}=\hat{A}^\dag}$, то оператор называется эрмитовым.

В стандартном базисе любой оператор $\hat{A}\in \mathcal{L}(\mathcal{H})$ можно представить разложением
\begin{equation}\label{}
\hat{A}\;=\; \sum\limits_{k,l=0}^{N-1}a_{kl}\ketbra{k}{l},\quad k,l=0,\ldots,N\!-\!1.
\nonumber
\end{equation}
Тогда требование эрмитовости оператора означает, что
\begin{equation}\label{}
a^\dag_{ij}= \bra{i}\hat{A}^\dag\ket{j}= \overline{\bra{j}\hat{A}\ket{i}}= \bar{a}_{ji}, \quad A^\dag=\bar{A}^{\:\!T},
\nonumber
\end{equation}
где $A$ --- матрица оператора $\hat{A}$. У любого эрмитова оператора существует ортонормированный базис (не единственный, если спектр оператора вырожден), состоящий из собственных векторов этого оператора. В таком базисе матрица оператора диагональна, а на главной диагонали стоят его собственные значения, причем все они вещественны.

При измерении состояния квантовой системы, т.е. при воздействии на нее, которое соответствует эрмитову оператору $\hat{A}$, система переходит в одно из собственных состояний оператора. Кроме того, постулируется, что с каждым эрмитовым оператором $\hat{A}$ связана некоторая физическая величина ${A}$, которая при таком переходе принимает значение, равное соответствующему собственному значению  оператора $\hat{A}$. Пусть система перед измерением находится в состоянии $\ket{u}$. Обозначим векторы собственного базиса оператора $\hat{A}$ индексом $\scriptstyle{A}$ и разложим $\hat{A}$ и $\ket{u}$ по этому базису:
\begin{equation*}
\hat{A}= \sum\limits_{k=0}^{N-1}a_{k} \ket{k}_{\!\scriptscriptstyle\!A}\!\bra{k}_{\scriptscriptstyle\!A}, \qquad
\ket{u}= \sum\limits_{k=0}^{N-1} u_{k}\ket{k}_{\!\scriptscriptstyle\!A}.
\end{equation*}
Тогда вероятность перехода квантовой системы из состояния $\ket{u}$ в состояние $\ket{k}_{\scriptscriptstyle\!A}$ при измерении равна $p_k= |u_{k}|^2$, а среднее значение физической величины ${A}$ равно
\begin{equation*}
  \langle{A}\rangle= \bra{u}\hat{A}\ket{u}= \sum\limits_{k=0}^{N-1}a_{k}p_k.
\end{equation*}

Матричные элементы $a_{kl}=\bra{k}\hat{A}\ket{l}$ являются, вообще говоря, комплексными числами, а базисные операторы $\ketbra{k}{l}$ не эрмитовы. В ${\mathcal{L}(\mathcal{H}_n)}$ существует особый эрмитов базис, который называется базисом Паули. Он строится на основе трех однокубитных эрмитовых операторов Паули и тождественного оператора (для краткости названия $\hat{\sigma}_0$ также относят к операторам Паули)
\begin{eqnarray}\label{}
\hat{\sigma}_0=\,\vphantom{\int} \ketbra00+\ketbra11, \!\!&\quad&\!\! \hat{\sigma}_1=\,\ketbra01+\ketbra10,
\label{sigma-1}\\
\hat{\sigma_3}=\,\ketbra00-\ketbra11,  \!\!&\quad&\!\! \hat{\sigma}_2=-i\ketbra01+i\ketbra10, \label{sigma-2}
\end{eqnarray}
Соответствующие эрмитовы матрицы называются матрицами Паули:
\begin{equation}\label{matr-Pauli}
\sigma_0=
\left(\!\!
\begin{array}{cc} 1&0 \\[4pt] 0&1 \end{array}
\!\!\right)\!,\quad
\sigma_1=
\left(\!\!
\begin{array}{cc} 0&1 \\[4pt] 1&0 \end{array}
\!\!\right)\!,\quad
\sigma_2=
\left(\!\!
\begin{array}{cc} 0&-i \\[4pt] i&\,0 \end{array}
\!\!\right)\!,\quad
\sigma_3=
\left(\!\!
\begin{array}{cc} 1&\,0 \\[4pt] 0&-1 \end{array}
\!\!\right)\!.
\nonumber
\end{equation}
Выделенность базиса Паули в ${\mathcal{L}(\mathcal{H})}$ проявляется в соотношениях
\begin{equation*}
\hat{\sigma}_k^\dag= \hat{\sigma}_k,\quad
\hat{\sigma}_k^2= \hat{\sigma}_0,\quad k=0,1,2,3,
\end{equation*}
\begin{equation*}
\hat{\sigma}_k\hat{\sigma}_l= i\:\!\mathrm{sgn}(\pi)\:\!\hat{\sigma}_m,
\end{equation*}
где тройка индексов ${klm}$ является перестановкой набора индексов 123, а $\mathrm{sgn}(\pi)$ --- знак перестановки. Отсюда также следует важное свойство антикоммутативности нетождественных операторов Паули:
\begin{equation*}
\big\{\hat{\sigma}_k,\hat{\sigma}_l\big\}=0, \quad k,l\in\{1,2,3\},\;\; k\neq l.
\end{equation*}

На основе базиса Паули в ${\mathcal{L}(\mathcal{H})}$ определяется базис Паули в ${\mathcal{L}(\mathcal{H}_n)}$. А именно, это базис
\begin{equation}\label{basis}
\big\{\hat{\sigma}_{K}\big\}_{K=0}^{N^2-1}= \big\{\hat{\sigma}_{k_1\ldots{}k_n} \big\}_{k_1,\ldots,k_n\,\in\, \{0,1,2,3\}},
\quad
\hat{\sigma}_{K}\!=\! \hat{\sigma}_{k_1\ldots{}k_n}=\!\! \hat{\sigma}_{k_1}\!\!\otimes\ldots\otimes\!\hat{\sigma}_{k_n},
\nonumber
\end{equation}
где $K$ --- десятичное представление $(k_1\ldots k_n)_4$ (числа по основанию 4).
Таким образом, вещественная размерность вещественного пространства эрмитовых операторов равна $N^2=4^n$, а произвольный эрмитов оператор $\hat{A}$ допускает разложение
\begin{equation}\label{}
\hat{A}\;=\; \sum\limits_{K=0}^{N^2-1}a_K\hat{\sigma}_{K}
\nonumber
\end{equation}
с вещественными коэффициентами. Важно, что операторы $\hat{\sigma}_{K}$ являются и эрмитовыми, и унитарными, т.е. ${\hat{\sigma}_{K}^\dag=\hat{\sigma}_{K},\, \hat{\sigma}_{K}\hat{\sigma}_{K}=\hat{\sigma}_{0}}$, и любые два базисных оператора или коммутируют, или антикоммутируют.

\section{Гамильтонианы}

В квантовой механике каждой изолированной квантовой системе сопоставляется некоторый эрмитов (самосопряженный) оператор, который  называется оператором Гамильтона или, короче, гамильтонианом. Гамильтониан определяет эволюцию системы, подчиняющуюся уравнению Шредингера
\begin{equation}\label{H}
i\hbar\,\partial_t\ket{u}= \hat{H}\ket{u},
\end{equation}
а также спектр ее возможных стационарных состояний, которые подчиняются стационарному уравнению Шредингера
\begin{equation}\label{E}
\hat{H}\ket{u}= E\ket{u}.
\end{equation}
Собственное значение $E$ гамильтониана называется энергией системы в соответствующем состоянии, причем собственное значение может быть вырожденным. В уравнениях~(\ref{H})  и~(\ref{E}) вектор состояния $\ket{u}$ принадлежит гильбертову пространству ${\mathcal{H}}$ квантовой системы и ${\hat{H}\in\mathcal{L}(\mathcal{H})}$. Спектр системы может быть непрерывным, когда энергия системы принимает значения в некотором интервале вещественных чисел, или дискретным, как конечным, так и --- теоретически --- бесконечно счетным. В данной диссертации рассматриваются только модели квантовых систем с конечным спектром. В этом случае весь набор собственных значений гамильтониана может быть получен (или проверен) экспериментально посредством измерений уровней энергии системы, то есть суммы кинетической и потенциальной энергий всех частиц, связанных с системой. 

Гамильтониан может принимать самые различные формы в зависимости от конкретных характеристик квантовой системы, в частности, от размерности гильбертова пространства, в котором она моделируется. Например, гамильтониан системы двух электронов в молекуле водорода без учета взаимодействия спинов в гильбертовом пространстве $L_2(\mathbb{R}^6)$ имеет вид
\begin{multline}\label{H2-0}
\hat {H}_{\!\scriptscriptstyle H_2}= -\frac{\hbar^2}{2m}\nabla^2_{\!\mathbf{r}_1}- \frac{\hbar^2}{2m}\nabla^2_{\!\mathbf{r}_2}+ \frac{e^2}{|\mathbf{r}_1-\mathbf{r}_2|}
\vphantom{\int\limits_a}\\
-\frac{e^2}{|\mathbf{r}_1-\mathbf{R}_1|}- \frac{e^2}{|\mathbf{r}_1-\mathbf{R}_2|}- \frac{e^2}{|\mathbf{r}_2-\mathbf{R}_1|}- \frac{e^2}{|\mathbf{r}_2-\mathbf{R}_2|}\,, \quad
\end{multline}
где $\mathbf{r}_1$ и $\mathbf{r}_2$ --- координаты электронов, а $\mathbf{R}_1$ и $\mathbf{R}_2$ --- координаты ядер, которые считаются неподвижными. Этот гамильтониан представляет из себя сумму операторов кинетической энергии электронов и сумму потенциальных энергий взаимодействия электронов с ядрами и между собой. Во многих частных случаях собственные векторы и спектр гамильтониана можно найти аналитически. Однако это трудно сделать даже для сравнительно простого гамильтониана~(\ref{H2-0}), в частности и потому, что в структуре волновой функции в $L_2(\mathbb{R}^6)$ требуется явно учесть принцип тождественности электронов и, следовательно, взаимодействие спинов: волновой функция должна быть антисимметрична относительно перестановки любых двух частиц или их спинов. Поэтому в квантовой химии реальную систему, т.е. электронную оболочку молекулы, моделируют системой кубитов.

Системы кубитов широко используются в квантовом моделировании самых разнообразных квантовых систем. Под термином квантовое моделирование подразумевают имитацию динамики реальной квантовой системы посредством более простой и искусственной квантовой системы, в данном случае, системы кубитов. На математическом языке это означает близость гамильтонианов обеих систем, что означает возможность отображения динамики одной квантовой системы на другую. В конкретных задачах гамильтониан системы кубитов удобнее всего рассматривать в виде разложения по базису Паули и записывать как
\begin{equation}\label{ham-pauli}
\hat{H}\;=\; \sum\limits_{K\in\mathcal{T}}h_K\hat{\sigma}_{K},
\end{equation}
где $K=k_1\ldots k_n$ --- строка Паули или ее десятичное представление (как объяснено в разделе~\ref{BPauli}), $\mathcal{T}$ --- некоторое множество строк Паули (или, что эквивалентно, десятичных индексов), а $h_K$ --- вещественные числа.

В следующем разделе~\ref{JW-sec} кратко описывается, как гамильтонианы квантовой химии отображаются на гамильтонианы подходящих систем кубитов. При этом реальный гамильтониан подвергается значительному упрощению, которое заключается в пренебрежении взаимодействиями малой интенсивности и пространственным движением ядер.

В данной работе рассматриваются квантовые системы, которые, во-первых, характеризуются как \textit{малоразмерные}. Это означает, что для эффективного описания реальных взаимодействий или, например, реализации квантовой цепи необходимо достаточно малое число (пусть $n$) кубитов. Во-вторых, в гамильтониане~\ref{ham-pauli} число слагаемых невелико, т.е. $|\mathcal{T}|\ll4^n$ и он представляют собой гамильтониан системы с небольшим количеством степеней свободы. Такие гамильтонианы называются \textit{разреженными} (в базисе Паули). Следует отметить, что число слагаемых в разложении разреженного гамильтониана по стандартному вычислительному базису может быть большим вплоть до $4^n$. На современном этапе развития квантовых технологий применение малоразмерных систем кубитов с разреженными гамильтонианами к моделированию молекулярных структур и к проектированию современных квантовых устройств можно назвать универсальным.



\section{Преобразование Йордана-Вигнера в\\ квантовой химии}
\label{JW-sec}

Основные фундаментальные взаимодействия в веществе задаются кулоновскими силами между электронами и ядрами, но в целом они слишком сложны и прямые математические расчеты с использованием соответствующих гамильтонианов в настоящее время невозможны. Поэтому в физике и в квантовой химии для описания различных свойств системы используются эффективные взаимодействия, которые имитируют реальные. Именно такой подход используется при изучении электронной структуры, поскольку даже такая частная задача, как нахождение энергии основного состояния  в терминах исходных гамильтонианов (для молекулы водорода --- это гамильтониан~(\ref{H2-0})), является чрезвычайно сложной. Поэтому исходный гамильтониан вначале подвергают процедуре вторичного квантования, выражая его через фермионные операторы рождения и уничтожения.

Гамильтониан молекулы водорода~(\ref{H2-0}) в методе вторичного квантования и с учетом спиновых взаимодействий имеет вид
\begin{equation}\label{H2-1}
\hat {H}_{\!\scriptscriptstyle H_2} = \sum_{i,j}g_{ij}a^\dag_i a_j + \frac{1}{2}\sum_{i,j,k,l}h_{ijkl}a^\dag_i a^\dag_j a_ka_l,
\end{equation}
где
\begin{equation*}
\begin{aligned}
g_{ij}&= \sum\limits_{\alpha,\beta=0,1}\int_{\mathbb{R}^3} \overline{\phi_i^{\,\alpha}(\textbf{r})}\! \left(-\frac{\nabla^2}{2}
-\frac{e}{|\mathbf{r}-\mathbf{R}_1|}- \frac{e}{|\mathbf{r}-\mathbf{R}_2|}\right)\! \phi_j^{\:\!\beta}(\mathbf{r}) \:\!\mathrm{d}^3r,\\
\\
h_{ijkl}&= \sum\limits_{\alpha,\beta,\gamma,\delta=0,1} \int_{\mathbb{R}^3\times\mathbb{R}^3} \frac{\overline{\phi_i^{\,\alpha}(\textbf{r}_1)} \overline{\phi_j^{\,\gamma}(\textbf{r}_2)} \phi_k^{\,\delta}(\textbf{r}_2)  \phi_l^{\,\beta}(\textbf{r}_1)}{|\mathbf{r}_1-\mathbf{r}_2|} \:\mathrm{d}^3r_1 \mathrm{d}^3r_2.
\end{aligned}
\end{equation*}

Чтобы сконструировать и смоделировать системы квантовой химии во втором квантованном представлении на квантовом компьютере, необходимо сопоставить операторы, действующие на неразличимые фермионы, с операторами, действующими на различимые кубиты. Метод кодирования - отображение фермионного пространства Фока в гильбертово пространство кубитов так, что каждое фермионное состояние может быть представлено состоянием системы кубитов. Существует несколько методов кодирования для решения данной задачи, но в данной работе мы рассмотрим метод преобразования Йордана-Вигнера.

В данном методе мы сохраняем номер заполнения спиновой орбитали в состоянии ${|0\rangle}$ или ${|1\rangle}$ кубита (незанятом или занятом, соответственно). Если рассмотреть данное явление более детально, получим следующую картину:

    \begin{equation}\label{}
    \begin{array}{cc}
    \vspace{0,2cm}
    |f_{M-1},f_{M-2},\cdots,f_0\rangle\rightarrow|q_{M-1},q_{M-2},\cdots,q_0\rangle, \\
    q_p=f_p\in\{0,1\}.
    \nonumber
    \end{array}
    \end{equation}

Операторы фермионного рождения и уничтожения увеличивают или уменьшают число заполнения спин-орбитала на единицу, а также вводя мультипликативный фазовый коэффициент.

    \begin{equation}\label{}
    \begin{array}{cc}
    \vspace{0,2cm}
    a_p|f_{M-1},f_{M-2},\cdots,f_0\rangle = \\
    \delta_{{f_p},1}(-1)\sum_{i=0}^{p-1}f_i|f_{M-1},f_{M-2},\cdots,f_p\oplus1,\cdots,f_0\rangle, \\
    \\
    a_p^{\dag}|f_{M-1},f_{M-2},\cdots,f_0\rangle = \\
    \delta_{{f_p},0}(-1)\sum_{i=0}^{p-1}f_i|f_{M-1},f_{M-2},\cdots,f_p\oplus1,\cdots,f_0\rangle,
    \nonumber
    \end{array}
    \end{equation}
где ${\oplus}$ обозначает сложение по модулю 2 такое, что ${0\oplus0=1}$ и ${1\oplus1=0}$. Фазовый член ${(-1)\sum_{i=0}^{p-1}f_i}$ обеспечивает обменную антисимметрию фермионов. Оператор заполнения спин-орбитали задаётся формулой:

    \begin{equation}\label{}
    \begin{array}{cc}
    \vspace{0,2cm}
    \hat{n}_i=a_i^{\dag}a_i, \\
    \hat{n}_i|f_{M-1},f_{M-2},\cdots,f_0\rangle=f_i|f_{M-1},f_{M-2},\cdots,f_0\rangle,
    \nonumber
    \end{array}
    \end{equation}
и подсчитывает количество электронов на заданной спиновой орбитали.

Кубитные отображения операторов сохраняют эти особенности и задаются следующей формулой:

    \begin{equation}\label{}
    \begin{array}{cc}
    \vspace{0,2cm}
    a_p=Q_p\otimes Z_{p-1} \otimes \cdots \otimes Z_0, \\
    a_p^{\dag}=Q_p^{\dag} \otimes Z_{p-1} \otimes \cdots \otimes Z_0,
    \nonumber
    \end{array}
    \end{equation}
где ${Q=|0\rangle\langle1|=\frac{1}{2}(X+iY)}$ и ${Q^{\dag}=|1\rangle\langle0|=\frac{1}{2}(X-iY)}$. Операторы ${Q}$ и ${Q^{\dag}}$ изменяют рабочий номер целевой спин-орбитали, в то время как последовательность операторов ${Z}$ восстанавливает фазовый коэффициент обмена ${(-1)\sum_{i=0}^{p-1}f_i}$ - данный процесс можно назвать вычислением чётности состояния. Используя метод Йордана-Вигнера, мы получим, что второй квантованный фермионный гамильтониан преобразуется в линейную комбинацию произведений однокубитных операторов Паули:

    \begin{equation}\label{}
    H=\sum_jH_jP_j=\sum_jH_j\prod_i\sigma_i^j,
    \nonumber
    \end{equation}
где ${h_j}$ - действительный скалярный коэффициент, а ${\sigma^j_i}$ представляет собой один из элементов ${I}$, ${X}$, ${Y}$ или ${Z}$, ${i}$ обозначает, на какой кубит воздействует оператор, в свою очередь ${j}$ обозначает член гамильтониана. Каждый член множества ${P_j}$ в гамильтониане обычно принято называть "строкой Паули"\:\!, а количество неидентичных однокубитных операторов Паули в данной строке называется "весом Паули". Чтобы ещё больше прояснить вторичное квантованное кодирование по методу Йордана-Вигнера, мы применим его к фиктивной системе. Как ранее указывалось, мы предполагаем, что в состоянии Хартри-Фока для этой системы оба электрона занимают ${|A\rangle}$ орбиталов. Мы храним данные о действиях спин-орбиталей ${|A_{\uparrow}\rangle}$, ${|A_{\downarrow}\rangle}$, ${|B_{\uparrow}\rangle}$, ${|B_{\downarrow}\rangle}$, которые мы используем как ${|f_{B_{\downarrow}}, f_{B_{\uparrow}}, f_{A_{\downarrow}}, f_{A_{\uparrow}}}$ с ${f_i=0, 1}$. Затем состояние Хартри-Фока задаётся формулой:

    \begin{equation}\label{}
    |\Psi_{HF}\rangle=|0011\rangle,
    \nonumber
    \end{equation}
в которой описанное состояние соответствует антисимметричному определителю Слейтера. В таком случае волновая функция ${s_z=0}$ равна:

    \begin{equation}\label{}
    |\Psi\rangle=\alpha|0011\rangle+\beta|1100\rangle+\gamma|1001\rangle+\delta|0110\rangle.
    \nonumber
    \end{equation}

Полученное состояние можно сравнить с первым квантованным отображением базисного набора. Работая в базисе Йордана-Вигнера, мы с лёгкостью можем увидеть преимущество квантовых компьютеров перед их классическими аналогами для решения задач химии. Волновая функция взаимодействия полной конфигурации содержит ряд определяющих факторов, которые экспоненциально масштабируются в зависимости от количества электронов, примерно как ${\cal{O}}$${(M^N)}$. Таким образом, для этого требуется объём памяти, который экспоненциально увеличивается в зависимости от размера системы. Однако, используя квантовый компьютер, мы можем вместо этого хранить полную волновую функцию конфигурационного взаимодействия (FCI - Full Configuration Interaction), использующая только ${M}$ кубитов. Регистр из ${M}$ кубитов может находиться в суперпозиции ${2^M}$ вычислительных базовых состояний. В базисе Йордана-Вигнера каждый определитель Слейтера, необходимый для волновой функции FCI, может быть записан как одно из этих базовых вычислительных состояний. Таким образом, квантовые компьютеры могут эффективно хранить волновую функцию FCI. Это также верно для других вторичных квантованных кодировок.

Основным преимуществом кодирования по Йордану-Вигнеру является его простота. Однако, в то время как занятость спин-орбитали сохраняется локально, чётность сохраняется нелокально. Последовательность Z-операторов означает, что фермионный оператор, сопоставленный кубитам, обычно имеет вес ${\cal{O}}$${(M)}$ операторов Паули, каждый из которых воздействует на другой кубит. Альтернативой сопоставлению Йордана-Вигнера, которое ещё не нашло особого применения в данной области, но о котором стоит знать, является кодирование по чётности. Этот подход сохраняет чётность локально, а номер занятости - нелокально. Мы используем кубит ${p^{th}}$ для хранения чётности первых ${p-}$режимов:
    \begin{equation}\label{}
    |f_{M-1}, f_{M-2}, \cdots, f_0\rangle\rightarrow|q_{M-1}, q_{M-2}, \cdots, q_0\rangle,\; q_p=\left[ \sum_{i=0}^pf_i \right] (\!\!\!\!\!\!\mod2).
    \nonumber
    \end{equation}


\section{Постановка задач квантовой оптимизации}

Мы сосредоточимся на методах уменьшения количества кубитов, необходимых для второго квантованного подхода, используя симметрию ${\mathbb{Z}_2}$. По методу Йордана-Вигнера и кодированию  число кубитов равно количеству спин-орбиталей. Однако, поскольку гамильтониан симметричен, волновая функция может храниться в меньшем Гильбертовом пространстве.

%####################################################
%################# Глава 2 ##########################
%####################################################

\chapter{Две модели малоразмерных систем}

%####################################################

\section{Молекула водорода}

Молекула водорода в квантовой химии есть самая простейшая молекула, состав которой лишён сложности и содержит всего 2 ядра атомов водорода и 2 электрона. В случае, если расстояния между протонами молекулы являются минимальными, то волновые функции сильно перекрываются. Законы квантовой механики и квантовой химии объясняют принципы создания связи между атомами в молекуле водорода, поскольку именно эти законы определяют всю структуру молекулы. Два атома водорода притягиваются и создают молекулу только в том случае, если спины электронов в атомах антипараллельны, и тогда энергия взаимодействия электронов достигает минимума. Но если же спины атомов водорода будут параллельны, молекула водорода попросту не сможет образоваться, так как между атомами на любых расстояниях будут действовать силы отталкивания.

Как показано в разделе~\ref{JW-sec}, гамильтонианы, полученные в подходе вторично квантованного базисного набора волновых функций (гамильтониан~(\ref{H2-1}) для молекулы водорода), всегда можно выразить в виде линейной комбинации тензорных произведений локальных (однокубитных) операторов Паули с помощью преобразования Йордана-Вигнера. В частности, явная форма гамильтониана молекулярного водорода с учетом взаимодействия спинов, полученная в работе~\cite{Du2022}, может быть представлена разложением
\begin{multline}\label{H2-2}
\hat{H}_{\!\scriptscriptstyle H_2}\, =\: -\,0.042\,\hat{\sigma}_{0000}+ 0.178\,(\hat{\sigma}_{3000}+ \hat{\sigma}_{0300})
\\
\,-0.243\,(\hat{\sigma}_{0030}+ \hat{\sigma}_{0003})+ 0.171\,\hat{\sigma}_{3300}+ 0.176\,\hat{\sigma}_{0033}\qquad\qquad \vphantom{\int}
\\
\,\;+0.123\,(\hat{\sigma}_{3030}+ \hat{\sigma}_{0303})+
0.168\,(\hat{\sigma}_{3003}+ \hat{\sigma}_{0330})\qquad\qquad\qquad
\\
+0.045\,(\hat{\sigma}_{2112}- \hat{\sigma}_{2211}- \hat{\sigma}_{1122} + \hat{\sigma}_{1221})\quad\vphantom{\overbrace{AA}}
\end{multline}
с коэффициентами в атомных единицах Хартри.



Матрица $H$ гамильтониана молекулы водорода в стандартном базисе также имеет сильно разреженный, но некомпактный вид (значения коэффициентов округлены)
\begin{equation}\label{matr-Pauli}
\small{\left(\!\!
\begin{array}{cccccccccccccccc}
0.80\!\!\!\!\!&\;0&\;0&\;0&\;0&\;0&\;0&\;0    &\;0&\;0&\;0&\;0&\;0&\;0&\;0&\;0 \\[4pt]
\;0&\!0.35\!\!\!\!\!&\;0&\;0&\;0&\;0&\;0&\;0    &\;0&\;0&\;0&\;0&\;0&\;0&\;0&\;0 \\[4pt]
\;0&\;0&\!0.35\!\!\!\!\!&\;0&\;0&\;0&\;0&\;0    &\;0&\;0&\;0&\;0&\;0&\;0&\;0&\;0 \\[4pt]
\;0&\;0&\;0&\!0.61\!\!\!\!\!&\;0&\;0&\;0&\;0    &\;0&\;0&\;0&\;0&\;0.18&\;0&\;0&\;0 \\[4pt]
\;0&\;0&\;0&\;0&\!\!\!-0.48\!\!\!\!\!&\;0&\;0&\;0   &\;0&\;0&\;0&\;0&\;0&\;0&\;0&\;0  \\[4pt]
\;0&\;0&\;0&\;0&\;0&\!\!\!-0.44\!\!\!\!\!&\;0&\;0   &\;0&\;0&\;0&\;0&\;0&\;0&\;0&\;0  \\[4pt]
\;0&\;0&\;0&\;0&\;0&\;0&\!\!\!-0.26\!\!\!\!\!&\;0   &\;0&\!-0.18\!&\;0&\;0&\;0&\;0&\;0&\;0  \\[4pt]
\;0&\;0&\;0&\;0&\;0&\;0&\;0&\!\!\!-0.49\!\!\!\!\!   &\;0&\;0&\;0&\;0&\;0&\;0&\;0&\;0 \\[4pt]
\;0&\;0&\;0&\;0&\;0&\;0&\;0&\;0   &\!\!\!-0.48\!\!\!\!\!&\;0&\;0&\;0&\;0&\;0&\;0&\;0   \\[4pt] \;0&\;0&\;0&\;0&\;0&\;0&\!\!-0.18\!&\;0&\;0   &\!\!\!-0.26\!\!\!\!\!&\;0&\;0&\;0&\;0&\;0&\;0  \\[4pt]
\;0&\;0&\;0&\;0&\;0&\;0&\;0&\;0   &\;0&\;0&\!\!\!-0.44\!\!\!\!\!&\;0&\;0&\;0&\;0&\;0   \\[4pt]
\;0&\;0&\;0&\;0&\;0&\;0&\;0&\;0   &\;0&\;0&\;0&\!\!0.49\!\!\!\!\!&\;0&\;0&\;0&\;0   \\[4pt]
\;0&\;0&\;0&\;0.18&\;0&\;0&\;0&\;0   &\;0&\;0&\;0&\;0&\!\!\!-1.08\!\!\!\!\!&\;0&\;0&\;0   \\[4pt]
\;0&\;0&\;0&\;0&\;0&\;0&\;0&\;0   &\;0&\;0&\;0&\;0&\;0&\!\!\!-0.36\!\!\!\!\!&\;0&\;0   \\[4pt]
\;0&\;0&\;0&\;0&\;0&\;0&\;0&\;0   &\;0&\;0&\;0&\;0&\;0&\;0&\!\!\!-0.36\!\!\!\!\!&\;0   \\[4pt]
\;0&\;0&\;0&\;0&\;0&\;0&\;0&\;0   &\;0&\;0&\;0&\;0&\;0&\;0&\;0&\!1.06
\end{array}    \!\!\right)}
\nonumber
\end{equation}
Стандартными методами линейной алгебры нетрудно найти наименьшее собственное значение (минимальное значение энергии) и соответствующий собственный вектор (основное состояние)
\begin{equation}\label{Emin-state}
E_{min}= - 1.136,\quad \ket{min}= 0.995*\ket{0011}- 0.105*\ket{1100}.
\nonumber
\end{equation}

Гамильтониан~(\ref{H2-2}) обладает очевидной симметрией --- инвариантностью относительно перестановок кубитов $\hat{\sigma}_{abcd}\mapsto\hat{\sigma}_{badc}$. Более того, с точностью до значений коэффициентов форма разложения~(\ref{H2-2}) (набор операторов Паули) сохраняется при перестановке $\hat{\sigma}_{abcd}\mapsto\hat{\sigma}_{cdab}$.  Отсюда можно сделать вывод, что обеими симметриями должны обладать также и собственные векторы гамильтониана. Кроме этого, операторы $\hat{\sigma}_{2}$ входят в 4-х кубитные операторы Паули только парами, поэтому все коэффициенты в разложении собственных векторов гамильтониана могут быть выбраны вещественными. На основании сказанного можно предложить несколько (по меньшей мере три) форм анзаца для проведения численных экспериментов в рамках вариационного квантового алгоритма. Классическая эмуляция алгоритма показывает, что наилучшие результаты дает четырехпараметрический универсальный анзац
\begin{equation}\label{ansatz}
\hat{U}(\bm\theta)= \mathrm{e}^{i\theta_4\hat{\sigma}_{1211}}\mathrm{e}^{i\theta_3\hat{\sigma}_{1112}} \mathrm{e}^{i\theta_2\hat{\sigma}_{1200}} \mathrm{e}^{i\theta_1\hat{\sigma}_{0021}},
\end{equation}
где $\theta_k\in[0,2\pi),\; k=1,2,3,4.$



\section{Моделирование GHZ и W состояний}

\section{Классическое моделирование квантовых систем}

%####################################################

\newpage
\addcontentsline{toc}{chapter}{\hspace{7mm} Заключение}
\section*{Заключение}

В работе получены следующие основные результаты:

%####################################################

\newpage
\addcontentsline{toc}{chapter}{\hspace{7mm} Литература}

\begin{thebibliography}{99}

\bibitem{Du2022}
Y. Du et al. \textit{Quantum circuit architecture search for variational quantum algorithms}. NPJ Quantum Information, \textbf{8}, 62, 2022 (Supplementary materials, (19)). \url{https://doi.org/10.1038/s41534-022-00570-y}

\bibitem{Tsirulev-Andre}
А.Н. Цирулёв, Э. Андре. \textit{Моделирование запутанных состояний в кластерах кубитов}. Физико-химические аспекты изучения кластеров, наноструктур и наноматериалов. Выпуск 14, c.\,342\,--\,349, 2022.\\
(\textit{https://physchemaspects.ru/2022/doi-10-26456-pcascnn-2022-14-342/})

\bibitem{Steeb-Hardy}
W. Steeb, Y. Hardy. \textit{Problems and solutions in quantum computing and quantum information}. World Scientific, 348 p., 2004.\\
(\textit{https://www.worldscientific.com/worldscibooks/10.1142/6077})

\bibitem{Tsirulev-Nikonov}
V.V. Nikonov, A.N. Tsirulev. \textit{Pauli basis formalism in quantum computations}. Mathematical Modelling and Geometry. Volume 8, No 3, pp.1-14, 2020.\\
(\textit{https://mmg.tversu.ru/images/publications/2020-831.pdf})

\end{thebibliography}

%####################################################

\newpage
\addcontentsline{toc}{chapter}{\hspace{7mm} Приложение}
\section*{Приложение}

\end{document} 